%Hier kommen die Strophen rein. 
%für Akkorde: \chort{Akkord}{Wort über dem der Akkord stehen soll}
%Strophenzahl/Refrain: \textbf{Refrain:} 

%\textbf{2.}\chort{d}{Heißer} \chort{a}{Tee} und \chort{C}{Funken}\chort{F}{glut}

\noindent\textbf{2.} \chort{G}{Vor} uns liegt die neue Stadt aus \chort{C}{Häu}sern Zelt an Zelt,\\
das \chort{D}{Le}ben in den Stadtteilen ver\chort{E7}{ändert} diese Welt.\\
Ich \chort{C}{se}he schon die Fahnen – \chort{G}{end}lich \chort{D}{bin} ich \chort{em}{da}!\\
Wir \chort{G}{lau}fen durch die Straßen spüren \chort{C}{Abenteuerluft},\\
Wir \chort{D}{sin}gen und wir lachen, riechen \chort{E7}{La}gerfeuerduft.\\
Lasst uns \chort{C}{trom}meln, lasst uns tanzen, steht für \chort{G}{uns}'re \chort{D}{Zu}kunft \chort{em}{ein}.\\[0.5em]

\pagebreak

\noindent\textbf{Refrain}\\

\noindent\textbf{3.} \chort{G}{In} Gemeinschaft leben wir, sind \chort{C}{für}einander da.\\
Mit \chort{D}{wa}chen Augen durch die Welt, hier \chort{E7}{wer}den Wunder wahr.\\
Auf den \chort{C}{Au}genblick vertrauen, offen \chort{G}{mit}ei\chort{D}{nan}der \chort{em}{sein}.\\
\chort{G}{Nun} den ersten Schritt zu wagen, \chort{C}{das} erfordert Mut.
Dazu \chort{D}{sind} wir alle hier zusammen, \chort{E7}{dass} sich etwas tut.
Um die \chort{C}{Welt} ein bisschen besser zu ver\chort{G}{las}sen \chort{D}{als} sie \chort{em}{war}.\\[0.5em]

\noindent\textbf{Refrain} Hey, hey, \chort{G}{hey}, wir geh'n zu\chort{D}{sam}men in die \chort{G}{Zu}kunft!\\
Hey, hey, hey, wir er\chort{C}{schaf}fen uns'rer \chort{G}{Träu}me Weg.\\
Wo geht es \chort{C}{hin}? In die neue \chort{em}{Stadt}.\\
Du hast die \chort{C}{Wahl}, d'rum gehe nun den nächsten \chort{G}{Schritt}\\
